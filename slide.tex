\documentclass[14pt]{beamer}

% font
\usepackage{fontspec}
\setmainfont[Ligatures=TeX]{IPAPGothic}
\usepackage{xeCJK}
\setCJKmainfont{IPAPGothic}
\newfontfamily{\liberation}{Liberation Sans}
\newfontfamily{\notosans}{Noto Sans}

% graphic
\usepackage{graphics}
\usepackage{graphicx}
\usepackage{color}
\usepackage{xcolor}
\usepackage{colortbl}
\definecolor{mygray}{rgb}{0.1, 0.1, 0.1}

% tikz
\usepackage{tikz}
\usetikzlibrary{automata}
\usetikzlibrary{arrows}
\usetikzlibrary{arrows.meta}
\usetikzlibrary{positioning}
\usetikzlibrary{intersections, calc}
\usetikzlibrary{decorations}
\usetikzlibrary{decorations.markings}
\usetikzlibrary{decorations.pathreplacing,angles,quotes}
\usetikzlibrary{fit}
\usetikzlibrary{math}
\usetikzlibrary{shapes}
\usepackage{pgfplots}
\usepackage{bchart}

% href
\usepackage{hyperref}
\hypersetup{
	colorlinks=true,
	linkcolor=cyan,
	filecolor=cyan,
	urlcolor=cyan,
	pdfnewwindow=true}

% beamer
\usepackage{bxdpx-beamer}
\usetheme{Boadilla}
\setbeamertemplate{navigation symbols}{}
\setbeamercovered{transparent}
\setbeamertemplate{frametitle}{%
	\vspace{0.1em}
	\usebeamerfont{frametitle}\insertframetitle%
	\par
	\rule[0.5\baselineskip]{0.9\paperwidth}{0.4pt}%
	\vspace{-0.5em}}
\setbeamertemplate{footline}{
	\hfill
	\usebeamercolor[fg]{page number in head/foot}
	\usebeamerfont{page number in head/foot}
	{\small \insertframenumber}
	\kern1em\vskip5pt
}
\setbeamercolor{footline}{fg=black,bg=black}

% itemize
\usepackage{enumitem}
\setitemize{itemsep=0.2em}
\setlength\leftmargini{20pt}
\setlength\leftmarginii{20pt}
\setlength\leftmarginiii{20pt}
\setlength\leftmarginiv{20pt}
\setlist[itemize,1]{label=$\color{blue}\bullet$}
\setlist[itemize,2]{label=$\color{orange}\triangleright$}
\setlist[itemize,3]{label=$\color{gray}\bullet$}
\setlist[itemize,4]{label=$\color{red}\triangleright$}
\setlist[itemize,5]{label=$\color{gray}\bullet$}
\setlist[itemize,6]{label=$\color{red}\triangleright$}
\setlist[itemize,7]{label=$\color{yellow}\bullet$}
\setlist[itemize,8]{label=$\color{pink}\triangleright$}
\setlist[itemize,9]{label=$\color{black}\bullet$}

% math
\usepackage{amsmath,amssymb,amsthm}
\usepackage{bm}

% other
\usepackage{caption}
\usepackage{cancel}
\usepackage{epigraph}
\usepackage{fancybox}
\usepackage{here}
\usepackage{makecell}
\usepackage{setspace}
\usepackage{scrextend}
\usepackage{svg}
\usepackage{ulem}
\usepackage{multirow}

\begin{document}

\begin{frame}
	\begin{center}
		{\huge \color{cyan} そうぇり もぷもぷ}
	\end{center}
\end{frame}


\begin{frame}
	\frametitle{soweli Mopumopu}
	
	\begin{itemize}
		\item Twitter上のトキポナボット
			\begin{itemize}
				\item \href{https://twitter.com/soweli_mopumopu}{@soweli\_mopumopu}
			\end{itemize}
	\end{itemize}
	
	\begin{figure}[H]
		\centering
		\includegraphics[width=10cm]{mopumopu.png}
	\end{figure}
\end{frame}

\begin{frame}
	\frametitle{soweli Mopumopu}
	
	\begin{itemize}
		\item 15分おきに何か言葉を発します
		\item 会話もできます たのしいね
	\end{itemize}
	
	\begin{figure}[H]
		\centering
		\begin{minipage}{0.45\linewidth}
			\includegraphics[height=2.7cm]{mopu1.png}
		\end{minipage}
		\begin{minipage}{0.45\linewidth}
			\includegraphics[height=2.7cm]{mopu2.png}
		\end{minipage}
	\end{figure}
\end{frame}

\begin{frame}
	\frametitle{なんの話をするの}

	\begin{itemize}
		\item soweli Mopumopuはどうやって動いてるのか?
			\begin{itemize}
				\item 実はすこし高度なことをしている
			\end{itemize}
		\item もぷもぷの技術について説明します
	\end{itemize}
\end{frame}

\begin{frame}
	\frametitle{トキポナとは}

	\begin{itemize}
		\item みんなが知ってるミニマル言語
			\begin{itemize}
				\item 語彙数: 120語
				\item 習得がかんたん たのしい!
				\item できる人が割と多い たのしい!
			\end{itemize}
		\item 知らない人のために
			\begin{itemize}
				\item \href{https://twitter.com/notolytos/status/1409484535151042568}{\small 初級トキポナ文法簡介}
				\item \href{https://www.youtube.com/watch?v=9C0YqTs4vB8}{\small 2分40秒で世界一簡単な言語を紹介して伝授する}
				\item \href{https://www.youtube.com/watch?v=wIFJfAhiPlE}{\small トキポナレッスン1「トキポナってなに」}
				\item \href{https://en.wikibooks.org/wiki/Updated\_jan\_Pije\%27s\_lessons}{\small jan Pije's lessons} 
					{\scriptsize (\href{https://github.com/stefichjo/toki-pona/blob/master/pije.md}{注})}
				\item \href{https://www.youtube.com/watch?v=2jRtYBaZGgQ}{\small 【日本語訳】toki pona li toki pona - トキポナソング}
			\end{itemize}
	\end{itemize}
\end{frame}

\begin{frame}
	\frametitle{トキポナをモデル化する}

	\begin{itemize}
		\item トキポナをコンピュータで扱いたい!
			\begin{itemize}
				\item コンピュータには言語がわからぬ...
			\end{itemize}
		\item ``文''を数理モデル化して,``文の良さ''を評価しよう!
	\end{itemize}
\end{frame}

\begin{frame}
	\frametitle{確率的言語}

	\begin{itemize}
		\item 文を「その文が起きる確率」として捉える
			\begin{itemize}
				\item 言語を確率$P$によって,$P(\text{文})$で表す
				\item $P(\text{toki pona li toki pona!})$とか
			\end{itemize}
		\item なにがうれしいの?
	\end{itemize}
\end{frame}

\begin{frame}
	\frametitle{確率的言語}

	\begin{itemize}
		\item 次の2つの文$A, B$はどちらが``良い''文だろうか?
			\begin{itemize}
				\item 文$A$: toki pona li toki pona.
				\item 文$B$: a akesi ala alasa ale.
					\begin{itemize}
						\item 明らかに$A$のほうがよい
						\item $P(A) > P(B)$となるはず
					\end{itemize}
			\end{itemize}
		\item 確率で言語を近似することができれば,文法・意味的に``良い文''・``悪い文''を定性的に評価できる
			\begin{itemize}
				\item チャットボットが作れる
				\item 機械翻訳などもこの考えを元に作られている
			\end{itemize}
	\end{itemize}
\end{frame}

\begin{frame}
	\frametitle{言語モデル}

	\begin{itemize}
		\item 文に確率を与えるモデルのこと
		\item つまり,
			\begin{itemize}
				\item 文$w_1^n = w_1 w_2 \cdots w_n$を入力して,
				\item $P(w_1^n)$を計算・出力する関数のようなもの
			\end{itemize}
		\item どのように計算する?
			\begin{itemize}
				\item そもそも入力の長さが文ごとに違うしつらい…
			\end{itemize}
	\end{itemize}
\end{frame}

\begin{frame}
	\frametitle{分解する}

	\begin{itemize}
		\item 確率論の乗法定理を用いて,$P(w_1^n)$を分解
	\end{itemize}
	\begin{align*}
		P(w_1^n)
			& = P(w_1, w_2, \cdots, w_n) \\
			& = P(w_1) P(w_2, w_3, \cdots, w_n | w_1) \\
			& = P(w_1) P(w_2 | w_1) P(w_3, w_4, \cdots, w_n | w_1, w_2) \\
			& = P(w_1) P(w_2 | w_1) P(w_3 | w_1^2) P(w_4^n | w_1^3) \\
			& = \prod_{i=1}^{n} P(w_i | w_1^{i-1})
	\end{align*}

	\begin{itemize}
		\item 1単語ずつ計算してかければ文の確率になる!
	\end{itemize}

\end{frame}

\begin{frame}
	\frametitle{$P(w_i | w_1^{i-1})$を計算する}

	\begin{itemize}
		\item $1$番目から$i-1$番目の単語がわかってて,$i$番目の単語が起こる確率
		\item $P(\text{li} | \text{toki pona}) > P(\text{wile} | \text{toki pona})$ 
		\item Mopumopuでは,ニューラルネットを用いている
			\begin{itemize}
				\item ニューラル言語モデルっていう
			\end{itemize}
	\end{itemize}
\end{frame}

\begin{frame}
	\frametitle{Transformerモデル}

	\begin{itemize}
		\item Mopumopuで採用したニューラルモデル
			\begin{itemize}
				\item 2017年にGoogleの人たちが考案した
					\begin{itemize}
						\item Google翻訳のなかみもこれ
					\end{itemize}
			\end{itemize}
		\item 次からのスライドで細かいことを説明します
	\end{itemize}
\end{frame}

\begin{frame}
	\frametitle{Transformerモデル}

\end{frame}

\begin{frame}
	\frametitle{どのように学習するのか?}

	\begin{itemize}
		\item モデルの確率分布を言語の確率分布に近づける
			\begin{itemize}
				\item モデルと言語の相対エントロピーを小さくする
			\end{itemize}
	\end{itemize}
\end{frame}

\begin{frame}
	\frametitle{言語のエントロピー}
\end{frame}

\begin{frame}
	\frametitle{相対エントロピー}
\end{frame}

\begin{frame}
	\frametitle{クロスエントロピー}
\end{frame}

\begin{frame}
	\frametitle{確率的勾配降下法}
\end{frame}

\begin{frame}
	\frametitle{Adamアルゴリズム}
\end{frame}

\begin{frame}
	\frametitle{自己回帰生成}
\end{frame}

\begin{frame}
	\frametitle{ランダムサンプリングは?}
\end{frame}

\begin{frame}
	\frametitle{top-pサンプリング}
\end{frame}

\begin{frame}
	\frametitle{PyTorchを使って学習・推論をする}
\end{frame}

\begin{frame}
	\frametitle{言語モデルでチャットボットをつくる}

	\begin{itemize}
		\item ちょっと無理がある
		\item すこし怪しいことをしている
	\end{itemize}
\end{frame}

\begin{frame}
	\frametitle{Twitter APIを使う}
\end{frame}

\end{document}

